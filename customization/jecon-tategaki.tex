%%#!platex -kanji=uft-8
%%#BIBTEX pbibtex -kanji=utf-8 jecon-tategaki

% Author:	       Shiro Takeda
% First-written:       <2002/11/02>
% Time-stamp:	       <2019-01-22 14:17:53 st>
%
%############################## Main #################################
\documentclass[10pt]{tarticle}

%% natbib.sty を使う。
\usepackage[longnamesfirst]{natbib}
\usepackage{mathpazo}
\usepackage{color}
\definecolor{MyBrown}{rgb}{0.3,0,0}
\definecolor{MyBlue}{rgb}{0,0,0.3}
\definecolor{MyRed}{rgb}{0.6,0,0.1}
\definecolor{MyGreen}{rgb}{0,0.4,0}
\usepackage[dvipdfm,
bookmarks=true,%
bookmarksnumbered=true,%
colorlinks=true,%
linkcolor=MyBlue,%
citecolor=MyRed,%
filecolor=MyBlue,%
pagecolor=MyBlue,%
urlcolor=MyGreen%
]{hyperref}

\setlength{\topmargin}{-20pt} 
\setlength{\textheight}{648pt}
\setlength{\oddsidemargin}{15pt}
\setlength{\textwidth}{440pt}

\bibpunct{\hskip-0.5em(}{)\hskip-0.5em}{}{}{}{}

% dvipdfmx の縦書き用オプション
\AtBeginDvi{\special{landscape}}

% \DeclareFontShape{JY1}{mc}{m}{n}{<-> s * [0.961] jis}{}
% \DeclareFontShape{JY1}{gt}{m}{n}{<-> s * [0.961] jisg}{}

%#####################################################################
%######################### Document Starts ###########################
%#####################################################################
\begin{document}

\verb|\bibliographystyle| に \verb|jecon-tategaki.bst| を指定したもの。
\vspace*{1em}

クラスファイルに \texttt{tarticle} クラスを利用している。さらに、プリアンプルで
\begin{verbatim}
        \bibpunct{\hskip-0.5em(}{)\hskip-0.5em}{}{}{}{}
\end{verbatim}
と設定し、引用部分の括弧を全角に変更し、さらにスペースを調整している。こ
うしないと引用部分で半角括弧が利用されるため、少し見た目がぶかっこうにな
る。

また、
\begin{verbatim}
        \AtBeginDvi{\special{landscape}}
\end{verbatim}
と設定し、用紙を landscape (横向き)にしている。
\vspace*{2em}

引用例を4つ。\citet{ishikawa03:_green_gas_emiss_contr_open_econom}、
\citet{takeda06:_cge_analy_welfar_effec_trade} は英語の論文。
\citet{oyama99:_mark_stru} と \citet{kuroda97jp:keo} は日本語の論文。

\nocite{*}

%% BibTeX スタイルファイルの指定。
\bibliographystyle{jecon-tategaki}

%% BibTeX データベースファイルの指定。
%
% 一個上のフォルダにある jecon-example.bib を指定。
\bibliography{../jecon-example}

\end{document}

%#####################################################################
%######################### Document Ends #############################
%#####################################################################
% </pre></body></html>
% --------------------
% Local Variables:
% fill-column: 80
% End:
