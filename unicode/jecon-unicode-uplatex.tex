%#!uplatex 
%#BIBTEX upbibtex jecon-unicode-uplatex.aux
% 
% このファイルは uplatex でコンパイルすることを前提で作成しています。
%
%############################## Main #################################

\RequirePackage{plautopatch}

%% jsarticle クラスを利用。
\documentclass[uplatex,dvipdfmx]{jsarticle}

\usepackage[T1]{fontenc}

\usepackage{natbib}
\usepackage{url}
\usepackage{amsmath,amssymb,amsthm}

%% 色を付ける.
\usepackage{graphicx}
\usepackage{color}
\definecolor{MyBrown}{rgb}{0.3,0,0}
\definecolor{MyBlue}{rgb}{0,0,0.3}
\definecolor{MyRed}{rgb}{0.6,0,0.1}
\definecolor{MyGreen}{rgb}{0,0.4,0}
\usepackage[dvipdfmx,bookmarks=true,%
bookmarksnumbered=true,%
colorlinks=true,%
linkcolor=MyBlue,%
citecolor=MyRed,%
filecolor=MyBlue,%
pagecolor=MyBlue,%
urlcolor=MyGreen%
]{hyperref}

% 以下の命令を入れておかないと \DH が定義されていないというエラーとなる。
\renewcommand{\DH}{\fontencoding{T1}\selectfont{\symbol{240}}}

%#####################################################################
%######################### Document Starts ###########################
%#####################################################################
\begin{document}

% uplatex では以下のコードが必要。これがないと欧文のユニコード文字が間延びする。
% \kcatcode`ç=15% not cjk character

\begin{flushleft}
 {\Large \textbf{upLaTeX (\texttt{uplatex}) を利用するケース}}
\end{flushleft}

\vspace{1em}

% This document contains English, Română, 日本語,
% \foreignlanguage{greek}{Ελληνικά}, and
% \foreignlanguage{russian}{русский язык}.

\section{使い方}

\subsection{TeX ファイルの書き方}

\begin{itemize}
 \item このファイル (\texttt{jecon-unicode-uplatex.tex}) を参考にしてください。
 \item \texttt{bib} ファイルの書き方、\texttt{jecon.bst} の使い方は
       \texttt{jecon-unicode-lualatex.pdf} を見てください。
\end{itemize}

\subsection{コンパイル}

\begin{itemize}
 \item upLaTeX でのコンパイル。
\begin{verbatim}
uplatex jecon-unicode-uplatex.tex
upbibtex jecon-unicode-uplatex
uplatex jecon-unicode-uplatex.tex               
uplatex jecon-unicode-uplatex.tex
dvipdfmx jecon-unicode-uplatex.dvi
\end{verbatim}
 \item つまり、\TeX のコマンドとしては \texttt{uplatex} を、BibTeX のコマンドと
       しては \texttt{upbibtex} を利用します。
 \item \texttt{uplatex} では DVI ファイルしか作成できませんので、最後に
       \texttt{dvipdfmx} で PDF に変換します。
\end{itemize}

\section{例}


\subsection{引用の例}

以下は引用の例です。

\citet{sample-test2},
\citet{sample-test},
\citet{oo100:_no_title},
\citet{oo99:_no_title},
\citet{Ryza2016},
\citet{Ryza15:_advan_analy_spark_patter_learn_data_scale},
\citet{40020418914},
\citet{120005678435},
\citet{120005614155},
\citet{Krey2014},
\citet{yamazaki13:_japan},
\citet{chang-2013},
\citet{takeda2013jecon},
\citet{Takeda2012a},
\citet{arimura-takeda2012},
\citet{matloff__2012},
\citet{Boswell-2012},
\citet{takeda2012_cge},
\citet{有村-蓬田2012},
\citet{森201206},
\citet{Takeda2011b},
\citet{takeda2011c},
\citet{40018847518},
\citet{Jaeger2011},
\citet{takeda10:_cge_analy_welfar_effec_trade},
\citet{40017004376},
\citet{Ades-2010-EnergyUseand},
\citet{松浦2010a},
\citet{2009yamasue502165},
\citet{ThoughtWorksinc.08},
\citet{Biker-2007-unemployment},
\citet{takeda06:_cge_analy_welfar_effec_trade},
\citet{2007yamasue482353},
\citet{bohringer2007measuring},
\citet{Ahman2007},
\citet{横溝2007},
\citet{Bohringer2006},
\citet{chuokankyo06jp:ccs},
\citet{瀧川2006a},
\citet{bouet06:_is_erosion_of_tarif_prefer_serious_concer},
\citet{BabikerRutherford-2005-EconomicEffectsof},
\citet{Mcconnell2005},
\citet{somusho04jp:2000io-kaisetsu},
\citet{Stokey2004},
\citet{ishikawa03:_green_gas_emiss_contr_open_econom},
\citet{brooke03:_gams},
\citet{Hattori02},
\citet{miyazawa02:_io_intr},
\citet{isikawa02jp:_env_trade},
\citet{Loschel2002},
\citet{ihori02:_japa_tax},
\citet{hardle_sweave:_2002},
\citet{Hattori01},
\citet{katayama2001},
\citet{Babiker2000525},
\citet{rutherford00:_gtapin_gtap_eg},
\citet{Hattori00},
\citet{a.___2000},
\citet{nakamura00:_excel_io},
\citet{fujita99jp:_spatial_econom},
\citet{Babiker-1999-KyotoProtocoland},
\citet{markusen99jp:trade_vol_1},
\citet{oyama99:_mark_stru},
\citet{Iregui-1999-EFFICIENCYGAINSFROM},
\citet{okura96jp:nihon-no-zeisei},
\citet{MaggiRodr'iguez-Clare-1998-ValueofTrade},
\citet{kuroda97jp:keo},
\citet{barro97jp},
\citet{shigen97:_energ_balan},
\citet{wong95:_inter_trade_goods_factor_mobil_},
\citet{ishikawa94:_revis_stolp_samuel_rybcz_theor_produc_exter},
\citet{brezis93:_leapf_inter_compet},
\citet{kiyono93:_regu_comp_1},
\citet{krugman91:_geogr_trade},
\citet{helpman91:_inter_trade_trade_polic},
\citet{krugman91:_is_bilat_bad},
\citet{iwamoto91jp:haito-keika},
\citet{nishimura90:_micr_econ},
\citet{内田90},
\citet{wang89:_model_therm_hydrod_aspec_molten},
\citet{ito85:_inte_trad},
\citet{lucas76:_econom_polic_evaluat},
\citet{imai72:_micr_2},
\citet{imai71:_micr_1},
\citet{milne-thomson68:_theor_hydrod},
\citet{samuelson67:_econo_found},
\citet{allais1953},
\citet{hiraga_2015},

\subsection{ユニコード文字の例}

\begin{itemize}
 \item ファイルでは記述されていても、PDFにしたときに表示されない文字(空白になっ
       たりバツ印になったりします)がいくつかあります。XeLaTeX、LuaLaTeX、
       upLaTeXのどれを使うか、どのような設定がされているか、どんなフォントがイン
       ストールされているかによって表示される文字が変わるようです。私の環境では
       LuaLaTeX を利用しているときが表示される文字が一番多いです。
 \item Hällo Wörld, Französische Sprache.
 \item 弌、丐、丕、个、丱、丶、丼、丿、乂、乖乘、亂、亅、豫、亊、舒、弍、于、亞,
       亟、亠、亢、亰、亳、亶、从、仍、仄、仆、仂
 \item ㈱, ㈲, ㈹, ㌔, ㌘
 \item ①②③④⑤⑪⑫⑬⑭⑮⑯㉑㊱㊲㊳㊴㊵㊶㊷㊸㊹㊺㊻㊼㊽㊾㊿, ⓰⓱⓲⓳⓴, ⓵⓶⓷⓸⓹
 \item 人名
       \begin{itemize}
	\item 草彅剛 vs 草なぎ剛
	\item 髙村薫 vs 高村薫
	\item 内田百閒 vs 内田百間
	\item 宮﨑あおい vs 宮崎あおい
	\item 元藤燁子 vs 元藤あき子
       \end{itemize}
 \item フランス語: René Descartes, né le 31 mars 1596 à La Haye en France, et
       mort le 11 février 1650 à Stockholm, est un mathématicien, physicien et
       philosophe français. Il est considéré comme l'un des fondateurs de la
       philosophie moderne.
 \item ドイツ語: Albert Einstein gilt als Inbegriff des Forschers und Genies. Er
       nutzte seine außerordentliche Bekanntheit auch außerhalb der
       naturwissenschaftlichen Fachwelt bei seinem Einsatz für
       Völkerverständigung und Frieden. In diesem Zusammenhang verstand er sich
       selbst als Pazifist, Sozialist und Zionist. 
 \item Wikipedia の言語名: Аҧсшәа, Acèh, Адыгабзэ, Afrikaans,
       Alemannisch, አማርኛ, Aragonés, Ænglisc, العربية, ܐܪܡܝܐ, مصرى, অসমীয়া,
       Asturianu, Aymar aru, Azərbaycanca, تۆرکجه, Башҡортса, Boarisch,
       Žemaitėška, Bikol Central, Беларуская, Беларуская
       (тарашкевіца)‎, Български, भोजपुरी, Bahasa Banjar,
       বাংলা, བོད་ཡིག, বিষ্ণুপ্রিয়া মণিপুরী, Brezhoneg, Bosanski, ᨅᨔ ᨕᨘᨁᨗ, Буряад,
       Català, Chavacano de Zamboanga, Mìng-dĕ̤ng-ngṳ̄, Нохчийн, Cebuano,
       Chamoru, ᏣᎳᎩ, کوردیی ناوەندی, Qırımtatarca, Čeština, Kaszëbsczi,
       Словѣньскъ / ⰔⰎⰑⰂⰡⰐⰠⰔⰍⰟ, Чӑвашла, Cymraeg, Dansk, Deutsch,
       Zazaki, Dolnoserbski, ދިވެހިބަސް, ཇོང་ཁ, Eʋegbe, Ελληνικά, Emiliàn e
       rumagnòl, English, Esperanto, Español, Eesti, Euskara, Estremeñu, فارسی,
       Suomi, Võro, Føroyskt, Français, Arpetan, Nordfriisk, Furlan, Frysk,
       Gaeilge, Gagauz, 贛語, Gàidhlig, Galego, Avañe'ẽ, गोंयची कोंकणी / Gõychi
       Konknni, ગુજરાતી, Gaelg, Hausa, 客家語/Hak-kâ-ngî, Hawai`i, עברית, हिन्दी,
       Fiji Hindi, Hrvatski, Hornjoserbsce, Kreyòl ayisyen, Magyar, Հայերեն,
       Interlingua, Bahasa Indonesia, Interlingue, Ilokano, Ido, Íslenska,
       Italiano, ᐃᓄᒃᑎᑐᑦ/inuktitut, La .lojban., Basa Jawa, ქართული,
       Qaraqalpaqsha, Taqbaylit, Адыгэбзэ, Kongo, Gĩkũyũ, Қазақша,
       Kalaallisut, ភាសាខ្មែរ, ಕನ್ನಡ, 한국어, Перем Коми,
       Къарачай-малкъар, Kurdî, Коми, Kernowek,
       Кыргызча, Latina, Lëtzebuergesch, Лезги, Limburgs, Ligure,
       Lumbaart, Lingála, ລາວ, لۊری شومالی, Lietuvių, Latgaļu, Latviešu,
       Мокшень, Malagasy, Олык марий, Māori, Baso Minangkabau,
       Македонски, മലയാളം, Монгол, Молдовеняскэ,
       मराठी, Кырык мары, Bahasa Melayu, Malti, Mirandés,
       မြန်မာဘာသာЭрзянь, مازِرونی, Dorerin Naoero, Nāhuatl, Napulitano,
       Plattdüütsch, Nedersaksies, नेपाली, नेपाल भाषा, Nederlands, Norsk nynorsk,
       Norsk bokmål, Novial, Nouormand, Sesotho sa Leboa, Diné bizaad, Occitan,
       Oromoo, ଓଡ଼ିଆ, Ирон, ਪੰਜਾਬੀ, Kapampangan, Papiamentu, Picard, Deitsch,
       पालि, Norfuk / Pitkern, Polski, Piemontèis, پنجابی, پښتو, Português, Runa
       Simi, Rumantsch, Română, Tarandíne, Русский, Русиньскый,
       Kinyarwanda, संस्कृतम्, Саха тыла, Sardu, Sicilianu, Scots, سنڌي,
       Sámegiella, Srpskohrvatski / српскохрватски, සිංහල, Simple
       English, Slovenčina, Slovenščina, Gagana Samoa, ChiShona, Soomaaliga,
       Shqip, Српски / srpski, Sranantongo, SiSwati, Sesotho, Seeltersk,
       Basa Sunda, Svenska, Kiswahili, Ślůnski, தமிழ், తెలుగు, Tetun, Тоҷикӣ,
       ไทย, ትግርኛ, Türkmençe, Tagalog, Tok Pisin, Türkçe, Татарча/tatarça,
       ChiTumbuka, Twi, Reo tahiti, Удмурт, ئۇيغۇرچە / Uyghurche,
       Українська, اردو, Oʻzbekcha/ўзбекча, Vèneto, Vepsän kel’,
       Tiếng Việt, West-Vlams, Volapük, Winaray, Wolof, 吴语, Хальмг,
       IsiXhosa, მარგალური, ייִדיש, Yorùbá, Vahcuengh, Zeêuws, 中文, 文言,
       Bân-lâm-gú, 粵語, IsiZulu,
\end{itemize}

% --------------------
% Local Variables:
% fill-column: 80
% coding: utf-8-dos
% End:


\nocite{*}

%%% BibTeX スタイルファイルの指定.jecon-unicode.bst を指定.
\bibliographystyle{jecon-unicode}

%% BibTeX データベースファイルの指定.
%
% jecon-example-reverse.bib は一個上のフォルダにあるものを利用する。
\bibliography{../jecon-example,unicode-example}

\end{document}
%#####################################################################
%######################### Document Ends #############################
%#####################################################################

% --------------------
% Local Variables:
% fill-column: 80
% coding: utf-8-dos
% End:

